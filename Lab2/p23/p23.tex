%%%%%%%%%%%%%%%%%%%%%%%%%%%%%%%%%%%%%%%%%%%%%%%%%%%%%%%%%%%%%%%%%%%%%%%%%%%%%%%%
% assignment_template.tex
% A template for assignments
% https://github.com/mhyee/latex-examples/
%%%%%%%%%%%%%%%%%%%%%%%%%%%%%%%%%%%%%%%%%%%%%%%%%%%%%%%%%%%%%%%%%%%%%%%%%%%%%%%%


% LaTeX Preamble
% Load packages and set options as needed
%%%%%%%%%%%%%%%%%%%%%%%%%%%%%%%%%%%%%%%%%%%%%%%%%%%%%%%%%%%%%%%%%%%%%%%%%%%%%%%%

% Set the document class to "article"
% Pass it "letterpaper" option
\documentclass[letterpaper]{article}

% We don't need the special font encodings, but still
% good practice to include these. See:
%
% http://tex.stackexchange.com/questions/664/why-should-i-use-usepackaget1fontenc
% http://dsanta.users.ch/resources/type1.html
\usepackage[T1]{fontenc}
\usepackage{ae,aecompl}
\usepackage{listings}
\usepackage{fancyhdr} 
\usepackage{float} 
\usepackage{graphicx}
\restylefloat{figure} 
\usepackage{hyperref}


\usepackage[applemac]{inputenc}
% Packages for the math proof in this example
\usepackage{amsmath}
\usepackage{amsthm}

% Set the margins
\newcommand{\margin}{2cm}
\usepackage[top=\margin,right=\margin,left=\margin,bottom=\margin]{geometry}

% For definitions
% Theorem Styles
\newtheorem{theorem}{Theorem}[section]
\newtheorem{lemma}[theorem]{Lemma}
\newtheorem{proposition}[theorem]{Proposition}
\newtheorem{corollary}[theorem]{Corollary}
% Definition Styles
\theoremstyle{definition}
\newtheorem{definici�n}{Definici�n}[section]
\newtheorem{example}{Example}[section]
\theoremstyle{remark}
\newtheorem{remark}{Remark}



% Use fancyhdr to define our own headers
\usepackage{fancyhdr}
%\usepackage{python}
\setlength{\headheight}{25pt} % Keeps LaTeX happy, takes care of some warnings
\pagestyle{fancy}

% Definitions to fill the header with
% EDIT THESE FIELDS
%%%%%%%%%%%%%%%%%%%%%%%%%%%%%%%%%%%%%%%%%%%%%%%%%%%%%%%%%%%%%%%%%%%%%%%%%%%%%%%%
\newcommand{\course}{  Sistemas Operativos \\ ACI 343}
\newcommand{\assignment}{Pr�ctica 2.3: Procesos e Hilos}
\newcommand{\id}{RUT:}
\newcommand{\name}{Nombre:}
\renewcommand{\date}{ \today}
%%%%%%%%%%%%%%%%%%%%%%%%%%%%%%%%%%%%%%%%%%%%%%%%%%%%%%%%%%%%%%%%%%%%%%%%%%%%%%%%

% Now define the header. Make the text bold.
% We'll get something like:
%
% 123456789             LaTeX 101
% J. Random Student   Assignment N      Today's Date
% --------------------------------------------------
%
% This layout is pretty simple, and should be enough for an assignment
% If you want more, you can consult the documentation
% http://www.ctan.org/tex-archive/macros/latex/contrib/fancyhdr/fancyhdr.pdf
\lhead{\textbf{\id\\ \name}}
\chead{\textbf{\course\\ \assignment}}
\rhead{\textbf{\includegraphics[scale=0.35]{udla} \\ \date}}

% Here is an example for customising the numbering
% It changes the first level of numbering to bolded (a), (b), (c), etc
\renewcommand{\theenumi}{\textbf{(\alph{enumi})}}
\renewcommand{\labelenumi}{\theenumi}
% Other options to play with are to change \theenumii, \labelenumii, and enumii for the second level of nesting,
% and so on to \theenumiv, \labelenumiv, and enumiv for the fourth level of nesting.
% The possible formats are \arabic (1, 2...), \alph (a, b...), \Alph (A, B...), \roman (i, ii...), and \Roman (I, II...)

% Begin the actual typesetting, by starting the "document" environment
%%%%%%%%%%%%%%%%%%%%%%%%%%%%%%%%%%%%%%%%%%%%%%%%%%%%%%%%%%%%%%%%%%%%%%%%%%%%%%%%
\usepackage{color}
\definecolor{mygreen}{rgb}{0,0.6,0}
\definecolor{mygray}{rgb}{0.5,0.5,0.5}
\definecolor{mymauve}{rgb}{0.58,0,0.82}

\lstset{ %
  backgroundcolor=\color{white},   % choose the background color
  basicstyle=\footnotesize,        % size of fonts used for the code
  breaklines=true,                 % automatic line breaking only at whitespace
  captionpos=b,                    % sets the caption-position to bottom
  commentstyle=\color{mygreen},    % comment style
  escapeinside={\%*}{*)},          % if you want to add LaTeX within your code
  keywordstyle=\color{blue},       % keyword style
  stringstyle=\color{mymauve},     % string literal style
}


\begin{document}


\section*{Ejercicios}


\begin{enumerate}

\item Tenemos el siguiente programa fork\_ping.py

\begin{lstlisting}[language=python]
import os
import sys

child_pid = os.fork()
if child_pid == 0:
	# child process
	os.system('ping -c 20 www.google.com > /ping.out')
	sys.exit(0)

print "En el padre, el pid del hijo es %d" % child_pid
pid, status = os.waitpid(child_pid, 0)
print "esperando volver, pid = %d, estado = %d" % (pid, status)
\end{lstlisting}

\begin{itemize}
\item Ejec�talo e intenta comprender qu� es lo que realiza.
\item Haz un programa, fork\_ps.py, tomando como muestra este mismo, en el que el padre realice un ps -Af y exporte su salida a un fichero procesos.txt en el home de tu usuario. \\

Mientras que el padre espera y, cuando ya ha realizado el hijo su tarea, cambia el nombre del archivo "procesos.txt" a "proc.txt".
\end{itemize}

\item Este programa, base\_thread.py, utiliza dos hilos para ejecuci�n de la funci�n "imprime".

\begin{lstlisting}[language=python]
import threading
import time

def imprime(cont, numhilo):
    while cont > 0:
        print "Hilo: %d Contando hacia atras %d " % (numhilo, cont)
        cont -= 1
        time.sleep(5)
 
t1 = threading.Thread(target=imprime, args=(5,1))
t2 = threading.Thread(target=imprime, args=(4,2))

t1.start(); t2.start()

t1.join(); t2.join()
\end{lstlisting}

\begin{itemize}
\item Ejecuta y comprende el programa.
\item Modifica el c�digo para que se ejecuten 4 hilos con contadores: 10, 4, 3, 7.

\end{itemize}

\item En el siguiente programa, shared\_value.py, se utiliza la variable x como global y hay dos hilos que la utilizan:

\begin{lstlisting}[language=python]
import time
import logging

x = 0           # x Compartido
def foo():
    global x
    for i in xrange(100000000): x += 1

def bar():
    global x
    for i in xrange(100000000): x -= 1

t1 = threading.Thread(target=foo)
t2 = threading.Thread(target=bar)
t1.start(); t2.start()
t1.join(); t2.join()   # Espera a que se completen
print x                # El resultado esperado es 0
\end{lstlisting}

\begin{itemize}
\item �Cu�l es la salida? �Por qu�?
\item Modifica el programa para que el resultado sea 0.

\end{itemize}

\end{enumerate}



 

\end{document}

